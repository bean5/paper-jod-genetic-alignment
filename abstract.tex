\clearpage
\begin {abstract}
%\textbf{}

Antonymy, or opposition in meaning, by nature creates a dichotomy.  Dictionaries and thesauruses tend agree when it comes to antonym listings for abstract notions such as adjectives, adverbs.  However, only some dictionaries will provide antonyms for nouns such as $boy$ \opp $girl$.  This is either because the antonyms are unknown, not widely accepted, or perhaps prescriptively inhibited---meaning that the definition of antonym is taken to apply to certain parts of speech only, thus excluding nouns (concrete or otherwise).  In a preliminary study (not published; xyz), it was found that antonymy is psychologically real for concrete signs. One example includes such as $boy$ \opp $girl$ (Bean 2012).  

There are studies regarding antonymy for adjectives \cite{Zhang}, but few regarding antonymy for concretes such as nouns.  Nevertheless, understanding antonymy is critical to linguistic areas such as semantic primitives (Wierzbicka 2009), metaphor (Lakoff and Johnson 1999), synonymy, and even second-language acquisition.  Hence this work.

Anonymous native speakers of English Qualtrics survey respondents (25 total, female and male, ages 18-32)
%\oppnospace 70/30 gender split
 were asked to provide an antonym for each of 10 words; half of words were either abstract (adjectives) or concrete (nouns).  The Student’s t-test reveals that the count of antonyms for abstracts tend to be fewer than those for concretes.  Results concluded significance [$t(7) = 3.6029, p < 0.0087$].  

%Interestingly, most common antonyms for zipper was button and Velcro.  The most common antonyms for $sky$ were $ocean$ and $ground$\footnotemark.

This enhances our understanding of the way that we perceive our world and the way in which we might describe it as we move through time.  It could be posited that knowing the relation of things would allow a speaker to get by with less. This is a type of compression that would probably be helpful to memory as well as second-language acquisition.  (For example, instead of remembering the word \textit {toaster}, one might opt for the \textit {small non-oven crisping device}\footnotemark.)  Furthermore, this lends some credence to semantic primitives (Wierzbicka 2009) since NOT\footnotemark~is a semantic primitive---a semantic unit which could allow a compression of up to 50\%.  Learning to get get by with less might occur during second language acquisition and would undoubtedly rely on the fact that antonyms exist for both concrete and abstracts.  

\footnotetext[1]{It would also interesting to classify the types of antonyms that exist for concrete abstract; however, this is left to future work.  Furthermore, understanding which features are most salient about an item seem to be a good indicator as to how to best go about deriving an opposite (e.g. for \textit{big house} the best opposite is probably \textit{small house} since height is so salient; for sky the salient feature is elevation, hence both the ocean or ground are seen as viable antonyms)}
\footnotetext[2]{Go ahead and ask people to name what you are describing and see if they come up with \textit{toaster}}
\footnotetext[3]{As an aside, the pairs that Wierzbicka names such as GOOD \opp BAD are redundant if NOT is thought of an a way to oppose an item.  Wierzbicka states that a semantic primitive can only be defined by itself.  Indeed, with NOT as a semantic primitive, BAD can be simply defined as NOT GOOD (the opposite of GOOD) so it is not a semantic primitive.  However, one might do the opposite with GOOD, defining it as NOT BAD.  Determining which of the two is the semantic primitive is outside the scope of this study and perhaps not as important as proving whether semantic primitives exist (since each are equally efficient as proposed semantic primitive)}

\end{abstract}

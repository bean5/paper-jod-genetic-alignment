\clearpage
\begin {abstract}
%\textbf{}

Antonymy, or opposition in meaning, by nature creates a dichotomy.  Dictionaries and thesauruses tend agree when it comes to antonym listings for abstract notions such as adjectives, adverbs.  However, only some dictionaries will provide antonyms for nouns such as $boy\sim girl$.  This is either because the antonyms are unknown, not widely accepted, or perhaps prescriptively inhibited---meaning that the definition of antonym is taken to apply to certain parts of speech only, thus excluding nouns (concrete or otherwise).  In a preliminary study \cite{bean_2012}, it was found that antonymy is psychologically real for concrete signs. One example is the pair $boy\sim girl$.  

There are studies regarding antonymy for adjectives \cite{Zhang}, but few regarding antonymy for concretes such as nouns.  Nevertheless, understanding antonymy is critical to linguistic areas such as semantic primitives \cite{Wierzbicka}, metaphor \cite{Lakoff_CogModels}, synonymy, and even second-language acquisition.  Hence this work.

Anonymous native speakers of English Qualtrics survey respondents (32 total, female and male, ages 18-32)
%\oppnospace 70/30 gender split
 were asked to provide an antonym for each of 10 words; half of words were either abstract (adjectives) or concrete (nouns).  The Student’s t-test reveals that the count of antonyms for abstracts tend to be fewer than those for concretes.  Results concluded significance:
 	\begin{quote}
 		$t(7.881)=3.55902, p < 0.00461577$ when including all responses

 		$t(8)=4.16415, p < 0.00157362$ when including only female responses
 	\end{quote}  

This enhances our understanding of the way that we perceive our world and the way in which we might describe it as we move through time.  

\end{abstract}

\section {Future Work}
%\textbf{}
There are many possibilities for future work concerning antonymy.  Future studies could be created to:
	\begin {enumerate}
		\item determine whether nominal antonyms differ from abstract antonyms (recall binarity, communicativity, and transitivity);
		\item determine how natives select antonyms;
		\item determine how non-natives select antonyms;
		\item determine what makes some syntactical constructions paradoxical while others are not;
		\item determine whether familiarity plays a role ($bull$ vs. $horse$ as best opposite of $cow$);
		\item determine whether priming has an effect such as in cases of $potato$ \opp $tomato$;
		\item compare results of this data with that of traditional non-gradable antonyms such as $boy$ \opp $girl$, $male$ \opp $female$;
		\item compare results of this data with that of traditional gradable antonyms such as $tall$ \opp $short$, $hot$ \opp $cold$;
		\item explore the creative responses that were unique or that were morphologically based such as $teacher$ \opp $learner$;
		\item and explore antonymy in other languages.  
	\end {enumerate}
A need for this future work is apparent.  More evidence of antonymy would prove that opposition is more prevalent in language than is at first apparent.  It might be that Wierzbicka is right—that semantic primitives exist and that opposition, a NOT element plays a major role.

Future work need not focus on only one of the above aspects at a time.  Studies could perform two or more of the above if desired.


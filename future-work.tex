\section {Future Work}
%\textbf{}
There are many possibilities for future work insofar as non-institutionalized antonyms go.  Future studies could be created to:
	•	determine whether societally-determined antonyms share characteristics with traditional antonyms, namely binarity, communicativity, and transitivity;
	•	determine whether familiarity plays a role (bull vs. horse as best opposite of cow);
	•	determine whether priming has an effect such as in cases of potato ~ tomato;
	•	compare results of this data with that of traditional non-gradable antonyms such as $boy$ \opp $girl$, $male$ \opp $female$;
	•	compare results of this data with that of traditional gradable antonyms such as $tall$ \opp $short$, $hot$ \opp $cold$;
	•	explore the creative responses that were unique or that were morphologically based such as $teacher$ \opp $learner$;
	•	and explore antonymy in other languages.  

A need for this future work is apparent.  More evidence of antonymy would prove that opposition is more prevalent in language than is at first apparent.  It might be that Wierzbicka is right—that semantic primitives exist and that opposition, a NOT element plays a major role.

Future work need not focus on only one of the above aspects at a time.  Studies could perform two or more of the above if desired.


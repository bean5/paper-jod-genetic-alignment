\section {Citations}
%\textbf{}
Bertocchi, Alessandra. (2003). Antonyms and Paradoxes.  Argumentation 17.1: 113-122.
Nuckolls, Janis B. (1996). Sounds like Life: Sound-Symbolic Grammar, Performance, and Cognition in Pastaza 
Quechua. Oxford studies in anthropological linguistics, 2. New York: Oxford University Press.
Lindner, Sue. (1982). What Goes Up Doesn’t Necessarily Come Down: The Ins and outs of Opposites.  
Lakoff, George.  (1999). Cognitive Models and Prototype Theory. Concepts: core readings.  
Merriam-Webster, Incorporated. (2012). Dictionary and Thesaurus – Merriam-Webster Online.  Available Online at 
http://www.merriam-webster.com/ Accessed on 12-12-12.
Dictionary.com LLC.  (2012). Dictionary.com.  Available online at http://www.dictionary.com/ Accessed on 
2012-12-12.
Brewer, Larry. (2010). Synonym.Org. Available online at http://www.synonym.org/ Accessed on 2012-12-12.
Englisch-hilfen.de. (2012). Opposites-Vocabulary-Learning English. Available online at 
http://www.englisch-hilfen.de/en/words/opposites1.htm Accessed on 2012-12-12.
Wierzbicka, Anna. (2009). Universal Semantic Primitives as a Basis for Lexical Semantics.  Folia Linguistica.  
Wilbur, Richard. (1921). “Opposites”.  New York: Harcourt Brace Jovanich. 
Zhang, Jian-li, Shu-zhen Jiang. (Jul 2004). English Gradable Antonyms: Implicit Comparison and Explicit 
Comparison. Zhejiang Daxue Xuebao (Renwen Shebhui Kexue Ban)/Journal of   Zhejiang University 
(Humanities and Social Sciences Edition). 34.4: 124-130

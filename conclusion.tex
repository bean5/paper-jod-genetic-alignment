\section {Conclusion}
\label{conclusion}

As seen in one-tailed \textit{t}-tests, significance was found.  The null hypothesis is rejected and $H_{1}$ is accepted at the p=0.01 level.  This was the case regardless of whether the responses for men were included.

This proves that although xyz do exist for nouns, xyz for nouns behaves differently from xyz of adjectives.  At first, it is suprising that this is the case---afterall, one would expect that we understand and comprehend notions as concrete as nouns more capably than we would abstracts.  However, something that is concrete is characterized in terms of more than 1 abstract.  In fact, that is what makes them not abstract.  When a native is asked to come up with an xyz for it, it could be posited that the native draws upon this set of characteristics and decides to toggle at least one of them.  Take \textit{shoe} as an example.  \textit{Shoe} can be characterized as the following:

\begin{itemize}
	\item used as clothing
	\item clothes one's feet
	\item used to keep feet warm \\
	\vdots
\end{itemize}

The reason that other characteristics are not included such as color or size is because this example is to show what the natives are emphasizing.  Indeed, a function (verb) is the only characteristic that is not an adjective, which makes nouns even more different from adjectives.

The native could draw one the most marked or important characteristic of \textit{shoe} and determine that a good way to create an xyz would be to simply come up with something that performs the xyz function.  However, since there is nothing well-known that keeps the feet not-warm, the best the native can do is describe the status of one's feet without \textit{shoe(s)}.  This seems to be the case as the most common response for the \textit{shoe} was \textit{barefeet/barefoot}.  

Interestingly, since more than 1 xyz existed for every noun, it seems that \textit{contraries} are less apparent.  Nouns are characterized in so many dimensions that each can be approached along a different dimension and oppose it from that direction.  This has implications for semantic primitives, of course.  Although a theoretical compression of up to 50\% was possible for adjectives by using the semantic primitive \textit{NOT}, simply negating a noun is a rarely-used method.  In fact, only one respondent used this technique for selecting xyz!

There is still a lot to learn concerning xyz---especially for nominal xyz.  This study only proves that the number of xyz for nouns is higher than that for adjectives; it does not prove \textit{why} this is the case.  
%\section {Introduction (and Inspiration/Motivation)}

The work of author attribution is applicable and researchable in both secular and non-secular fields. In this study, we focus on an application of textual alignment between secular documents. Although we do not detect imitations, we do align them. 

Our primary, or imitable documents, include \textit{The Holy Bible KJV}, \textit{The Holy Bible JST} (Joseph Smith Translation), \textit{The Book of Mormon}, \textit{The Doctrine \& Covenants}, and \textit{The Pearl of Great Price} as sources. Secondary documents---those that attempt to imitate the primary documents---are composed of religious discourse texts: 10 volumes of the Journals of Discourse. %``The Journal of Discourses is not an official publication of The Church of Jesus Christ of Latter-day Saints. It is a compilation of sermons and other materials from the early years of the Church, which were transcribed and then published. It included some doctrinal instruction but also practical teaching, some of which is speculative in nature and some of which is only of historical interest.'' (http://www.lds.org/topics/journal-of-discourses) The discourses were last published in 1886. 
Using a database of pre-aligned imitation alignments, imitation class, and scripture imitation passages, we compare various genetic algorithms in terms of error rate. %and perhaps other metrics 
%[For error measures: We focus on only the right position of the alignments since we aim to insert imitations where it seems most appropriate for them.]
We then determine which of all the genetic algorithms have the least error for each type of imitation.

After determining which version of genetic algorithm works best for each type of imitation, we use the meta-data of the alignments (length, number of matches, number of indels, and number of replaces, etc.) and target (gold standard alignment positions) as input to the training of a neural network. %We found that there were xyz clusters, each corresponding to exact quote, non-exact, paraphrase, or allusion. This empirically demonstrates that given just xyz algorithms, we can discern between types of attribution that is called for. It is not outside the realm of possibilities for future work to use this technique to classify work at run-time before making a conclusion as to the badness of the imitation attempt.
According to overall error rates, neural networks---both linear and non-linear---outperformed all genetic alignment algorithms. %This was also true with respect to each imitation type.

This study aims to prove the following hypothesis:
	\begin{quote}
		$H_{1}:$ The error of any of our genetic algorithms is better than the baseline error of 232.

		$H_{2}:$ Given output and meta-data of genetic alignments, a neural network can be trained to produce lower error rates on imitation alignments than any of them alone as proven by the error of 10-fold cross validation.

		%$H_{3}:$ Given the meta-data of the text, we can classify the type of imitation with xyz\% accuracy.		
	\end{quote}

%Herein we also show the classification confusion matrix for type of imitation, which is not revelatory, but informative nonetheless.  [Future work: also show the meta-data profile, including a word-length histogram, word-token-length histogram, etc.].  

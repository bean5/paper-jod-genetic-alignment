

\subsection {Side A.} Wilbur in his book “Opposites” mentions for various words.  Interestingly, after each keyword and opposite, he often decides to list another opposite (Wilbur 1921).  He often does this comically by playing off ambiguities such as word senses (e.g. bat as an animal or as a sporting instrument).  Occasionally, on a more serious note, he lists multiple opposites for words which are not ambiguous, such as earth.  Wilbur accepts the possibility of opposites for non-modified nouns and in some cases agrees with the possibility of more than 1 opposite.  His opposite for earth is sky or heaven; squash, bean; hat, shoes; present, past or future.

Lindner works to diffuse the confusion surrounding directional opposites.  She says “The opposition relation is built squarely on a dichotomy or binary contrast of some sort.  That is, iron, gold, silver, etc., may make up a contrast set or multiple taxonomy, but typically only members of a binary taxonomy are called opposites.  Furthermore, the notion of opposition is based on contrast within similarity (Lyons 1977:286)—that is, there must be some reason to compare two lexical items, so that we would want to call man and woman opposites, but not rose and pig” (Lindner 1982). Lindner agrees that boy \opp girl (concrete nouns), female \oppmale (abstract nouns/adjectives) are acceptable binary contrast sets.  However, since words are being generated as language changes, perhaps this number of binary pairs should be fluctuate and change.  Lindner might accept laptop \opp desktop, Mac \opp PC, or mountain \opp valley. 

In prototype theory, nouns can be grouped and classified.  Visually, they can be represented as being more or less distant from the core idea that they represent (Lakoff 1999).  For example, the word bird is a class of animals for which robins are good examples, but emus are not.  This theory might explain the existence of antonyms for non-modified nouns by arranging nouns in a multi-dimension arena, with various connection types connecting nouns to other nouns.  One connection type might link characteristics in common, while another connection type might denote dissimilarity.  Locating the best opposite for a given noun may be as simple as traversing connections or pathways in a certain way.  Perhaps one could traverse the pathways a given number of steps in certain directions; perhaps one could detect which words are mostly similar but somewhat different by using the using the number of like- and unlike -connections.  This would allow for the detection of contrast within similarity that Lindner mentioned.  

As mentioned in 1.3.1., Wierzbicka’s framework of semantic primitives contains the ability to negate.  Assuming the existence of semantic primitives, antonymy beyond the well-accepted antonymy for abstracts is not extreme.  Meaning itself depends on it.

\subsection {Side B.} Admittedly, Lindner seems to prefer that the term opposite maintain a strict traditional usage rather than a usage which can be applied to societally-determined antonyms for non-modified nouns.  Perhaps only gender nouns are acceptable to her.  It is difficult to tell since she only lists man \opp woman and *rose \opp pig.  

Thesauri or dictionaries that list synonyms and antonyms readily do so for adjectives and gender nouns, but not for non-modified nouns, except indirectly.  Some online dictionaries and synonym/antonym finders are described below.
	•	http://www.merriam-webster.com/thesaurus—does not list opposite for binary concrete nouns such as girl/boy, but does for female/male but does for adjectives such as hot/cold.
	•	http://dictionary.reference.com/—does not list opposite for binary girl/boy nor for female/male, but does for hot/cold.
	•	http://www.synonym.org/—does not list opposite for binary girl/boy nor for female/male, but does for hot/cold.

Only some dictionaries allow for girl/boy or female/male, although listings for adjectival antonyms hot/cold are acceptable.  This dearth of listings of concrete antonyms demonstrates that institution has not accepted the possibility of antonyms for non-modified nouns. 

The only other opposition against the possibility of antonyms for non-modified nouns is the overwhelming amount of studies performed for traditionally institutionalized sets of opposites and lack of studies on antonyms for non-modified nouns.  
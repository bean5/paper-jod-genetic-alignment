\subsection {Categorizing Results} Using even this small set of antonym pairings, one can categorize their type.  The following list includes some words whose data is not shown above but for which salient best opposites exist.  Assuming antonymy can be a continuum, the second most salient words are also included below.
	\begin {enumerate}
		\item spatial contrast
			\subitem $sky$ \opp $ground$
		\item functional complementation (performing intended use in another manner) 
			\subitem $pencil$ \opp $pen$
			\subitem $potato$ \opp $steak$, $meat$
			%\subitem $dog$ \opp $cat$ (as pets)
			%\subitem $dog$ \opp $goldfish$ (as pets)
			\subitem $zipper$ \opp $Velcro$
			\subitem $zipper$ \opp $button$
			\subitem $carpet$ \opp $hardwood (floor)$
		\item functional opposition (opposing intended use; neutralize each other’s function) 
			%\subitem $pencil$ \opp $eraser$
			\subitem $zipper$ \opp $unzip$
		\item mutual exclusion (binary contrast sets only)
			\subitem $carpet$ \opp $hardwood (floor)$
			\subitem $robot$ \opp $human$
			%\subitem $teacher$ \opp $student$
		\item rhyme pairs*
			\subitem $potato$ \opp $tomato$
	\end {enumerate}

This is merely a rough list.  Future work might reveal a more complex hierarchy, just as a hierarchy exists for tak.  

The number of categories for these words is larger than that which exists for institutionalized words.  This lends credence to the belief that the process of creating opposites is being applied more abstractly to these words than to the traditionally accepted words. Rather than only having gradable and non-gradable antonyms alone, there seems to be a preponderance of antonym types.  Furthermore, this lends credence to the possibility that there is a tak-like process that allows for antonyms at the level of non-modified nouns to apply more liberally in spite of being less institutionalized.
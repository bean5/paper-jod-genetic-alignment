\section {Methodology}
%\textbf{}
An online survey was created consisting of 3 phases: demographic, free-form antonym decisions (10 possible), and feedback.  The order of non-demographic/non-feedback questions was randomized.  A preliminary survey was conducted to test for viability and probe for acceptable length.  The initial number of free-form antonym decision questions presented to the respondents was 10.  A lower number, 7, was decided upon after receiving feedback from initial respondents.  

\subsection{Survey Distribution.} The survey was distributed first by email, then listed on the author’s Facebook account to his friends.  It was then reposted on Facebook by two volunteers, 1 male and 1 female, thus bringing it to a wider audience.  

\subsection{Survey Format and Question Types.} Demographic questions requested gender, approximate age, and whether respondent’s native language was English (Yes/No question).  Only responses from native speakers of English will be analyzed here. 

The survey consisted of two question types in separate sections. In the first section, respondents were presented with a keyword. Then they were to give two antonyms in a free-form style. The respondent’s task was to provide the best and second-best opposite possible.  They were instructed to provide institutionalized opposites if they themselves believed them to be the best opposites possible. In the case that they could not come up with a second-best opposite, they were instructed to leave it blank, write “none”, or write “N/A” (meaning “not applicable”).  

\subsection{Survey Smoothing \& Normalization.} It was not uncommon to receive responses that varied only in terms of capitalization, plurality, or part of speech (e.g. chair, Chairs; alive, living).  These were counted to be semantically the same.  The motivation for this is not only because it makes sense but also because it seemed to confound the study.  What we want to prove is that the count of opposites for concretes tends to be higher than that for abstracts even without including the possibility of pluralization. 

Although this decreases the total count of opposites, this mainly impacted the number of opposites for concretes which makes the results in this survey more conservative than they might otherwise be.  Thus word groups such as alive and living are referred to collectively as alive/living. 
